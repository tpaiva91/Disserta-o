\chapter{Introdução}\label{chap:intro}

Algumas referências para se ver a ordenação  \cite{yaacoub2012}. Aqui outra ainda \cite{etsitr102732}.

Deve-se ler Strunk~\cite{strunk2007elements}

\ac{HTTP} é um protocolo como diz em~\cite{test2000} e também na secção~\ref{chap:stat}.

  Temos aqui o Clausen \cite{Clausen2003}. E para ver várias \cite{yaacoub2012, etsitr102732, strunk2007elements}

Deve-se acrescentar os acrónimos no ficheiro \texttt{acros.tex} e ordená-los alfabeticamente nesse ficheiro.


\section{Section example}
\lipsum[1-6]
\section{Second Section example}
\subsection{SubSection example}
\lipsum[7-10]

