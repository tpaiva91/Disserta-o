\chapter{Introdução}\label{chap:intro}


A diabetes, também conhecida por \textit{diabetes mellitus}, é uma doença crónica bastante comum, conhecida por fazer com que os seus portadores tenham níveis de glicose (açúcar) no sangue mais elevados que o normal. À concentração de glicose no sangue dá-se o nome de glicemia. Isto deve-se ao facto de o pâncreas não funcionar da forma devida ou nem sequer funcionar, de todo. Antes de nos aprofundarmos sobre a doença em si, eis alguns factos preocupantes:

Segundo a \ac{IDF} em 2015, cerca de 415 milhões de pessoas tinham diabetes. Em 2040, se continuar ao mesmo ritmo, este número aumentará para 642 milhões~\cite{atlas} e, de acordo com a \ac{OMS}, em 2030 a diabetes será a sétima causa de morte no planeta~\cite{who}.

Como se pode perceber, esta doença afeta muita gente e a tendência é para piorar. Por isso mesmo, torna-se cada vez mais importante conseguir adiar ou prevenir o seu aparecimento, que nem sempre é possível. O problema é que a diabetes não tem cura e portanto é fundamental que um paciente diabético tenha um tratamento adequado, sendo que o objetivo é manter os níveis de glicemia mais ou menos constantes, e dentro de intervalos considerados normais.
No entanto, não existe uma forma de tratamento padrão que possa ser aplicada a todos os doentes diabéticos. Além do tratamento médico, como a insulina ou medicamentos, há outros fatores que impactam, de alguma forma, a quantidade de glicose no sangue, como por exemplo o exercício, doenças ou o tipo de alimentos que se ingere. 
Nem toda a gente tem as mesmas rotinas e, portanto, um tratamento que seja eficaz num paciente pode não ser noutro. É por isso importante que os pacientes diabéticos tenham um tratamento personalizado, de acordo com as suas características e rotinas. Normalmente, o tratamento de um paciente diabético passa por um plano elaborado conjuntamente pelos seus médicos endocrinologista e nutricionista. Este plano será sempre feito tendo em conta o paciente, pelo que é um plano personalizado de acordo com as necessidades e rotinas do mesmo. Isto é a base de um conceito que será abordado no próximo capítulo, medicina personalizada.

Ainda no tratamento da doença, a parte da alimentação e rotinas é bastante importante. A diferença entre fazer sempre as mesmas refeições a horas certas ou não ter qualquer tipo de rotina neste aspeto pode ser a diferença entre valores normais ou descontrolados. Uma medição frequente, para que o paciente vá controlando os seus níveis de glicemia e tomar ações, se necessário, é um fator importante para a estabilização dos valores de glicose. Por medição frequente entenda-se uma medição várias vezes ao dia, porque o valor de glicemia pode ter oscilações rápidas e, portanto, medições com curtos intervalos de tempo detetarão mais rápido estas oscilações.
De facto, um controlo rigoroso dos níveis de glicose pode minimizar ou até prevenir as consequências da diabetes, como vamos ver na próxima secção.



\section{Motivação}

Como mencionado, o controlo dos níveis da glicose, através de medições frequentes, é um fator importante para o aumento da qualidade de vida do doente diabético. Um estudo levado a cabo entre 1983 e 1993 comprova isto mesmo: participaram 1441 voluntários e nesse período de 10 anos tiveram um controlo intensivo da glucose que lhes permitia ter valores próximos dos normais. O controlo intensivo era feito aumentando o número de medições diárias, aumentando o número de injeções de insulina ou com o uso de bomba, ajustando sempre o valor de insulina de acordo com a alimentação e exercício, seguindo uma dieta e plano de exercícios e fazendo visitas mensais ao centro de saúde para avaliar o progresso. O estudo concluiu que um controlo intensivo da glucose levou a uma redução em pelo menos 50\% de risco de doenças renais, oculares ou do sistema nervoso~\cite{edit}. Quando se fala em controlo intensivo da glicemia, não há um número mínimo de medições diárias definido, mas \textit{websites} da especialidade referem que devem ser feitas entre 4 a 8 medições diárias de glicemia para diabéticos tipo 1~\cite{measures1, measures2}.
Ou seja, apesar de ser uma doença crónica, é possível aumentar a qualidade de vida dos pacientes diabéticos, desde que tenham os cuidados acima mencionados. 
Como é possível perceber, a medição e registo da glicose são processos fundamentais para um bom tratamento da doença. No passado, esse registo tinha que ser feito em papel, que tem como inconveniente o facto de ser passível de se perder ou se tornar rapidamente confuso e extenso. No entanto, hoje isso já não se verifica.
A tecnologia evoluiu de tal forma que foram criados dispositivos com o propósito de medir e registar os níveis de glicemia. Mas os próprios telemóveis, que são cada vez mais baratos e melhores, tornaram-se inteligentes e são hoje ferramentas poderosas que fazem muito mais do que apenas ligar a alguém ou enviar mensagens. 
Um \textit{smartphone} pode servir para fotografar, jogar ou até ouvir música, mas pode ser usado também como uma ferramenta para o nosso bem-estar, o que se verifica, havendo aplicações destinadas à saúde. A motivação para este trabalho foi a possibilidade de juntar duas áreas diferentes, a saúde e a tecnologia, para desenvolver uma ferramenta que possa ter um impacto positivo na vida dos doentes diabéticos. 

\section{Projeto}


Esta dissertação integra-se no projeto ``Smart Diabetes Self-Management'' que conta com uma aplicação para Android chamada ``My Diabetes''. Esta aplicação visa oferecer aos seus utilizadores uma alternativa para o registo das medições de glicose, e facilitar a visualização desses mesmos registos, através de gráficos ou em forma de lista. A aplicação será descrita mais detalhadamente no capítulo 4. 

O trabalho proposto nesta dissertação foi o de desenvolver novas funcionalidades para a aplicação, dando-lhe alguma ``inteligência''. Foi proposto desenvolver um sistema que, através da análise dos dados inseridos por cada utilizador ao longo do tempo, fosse capaz de aprender as rotinas para que pudesse gerar avisos ou conselhos face a situações anormais ou até mesmo descobrir padrões que levem a resultados indesejados. Ao descobrir uma destas situações e alertar o utilizador para a mesma, estará a contribuir para que este consiga melhorar o seu controlo da glicemia.

Para desenvolver esta nova funcionalidade, foi necessário obter dados de pacientes insulino-dependentes. Deste modo, em parceria com o Hospital de São João do Porto, foi levada a cabo uma sensibilização dos doentes para utilizarem a aplicação de forma voluntária, sendo que, no futuro, serão os utilizadores da aplicação os mais beneficiados.
A utilização voluntária da aplicação por parte dos pacientes tem diversos objetivos: 1) obter \textit{feedback} da aplicação em si, como críticas ou sugestões; 2) poder construir \textit{data sets} de registos glicémicos num espaço temporal, algo escasso na \textit{web}. Esta parte de obtenção e análise dos dados é fundamental uma vez que permite ter mais conhecimento do tipo de dados que vão ser analisados, bem como o tipo de padrões ou regras que podem ser descobertas. Desta forma, será possível saber o que é útil ou não, para que a aplicação apenas mostre o que realmente for importante. 

A análise é feita aos dados que os utilizadores inserem na aplicação e enviam. Mais informações tais como os dados registados e recolhidos ou o processo de participação no estudo serão abordados com mais detalhe no capítulo 5.


\subsection{Objetivos}

O objetivo desta dissertação é analisar dados reais pertencentes a doentes diabéticos e desenvolver um sistema capaz de gerar regras e mostrar avisos ou conselhos a partir dos dados. Este objetivo não é único, sendo que os outros são:

\begin{itemize}
	\item Obter dados de registos glicémicos através da participação de voluntários diabéticos;
	\item Fazer diferentes tipos de análises estatísticas sobre esses dados;
	\item Analisar os dados para reconhecimento de padrões ou anomalias;
	\item Criar regras a partir da análise de dados;

\end{itemize}


\subsection{Contribuição}

Este trabalho tem algumas contribuições, que são:

\begin{itemize}
	\item revisão da literatura em aplicação de tecnologias inteligentes ao controlo da diabetes;
	\item recolha de dados e criação de \textit{data sets} de registos diabéticos;
	\item análise estatística de registos diabéticos;
	\item criação de regras personalizadas para cada utilizador.
	
\end{itemize}


\section{Organização}

Esta dissertação está organizada da seguinte forma: no Capítulo~\ref{chap:concepts} serão apresentados alguns fundamentos e conceitos relativamente à diabetes e às tecnologias que irão ser utilizadas.
No Capítulo~\ref{chap:stat} faremos uma revisão das tecnologias já aplicadas à saúde e, mais especificamente, à diabetes. Será feita uma comparação entre algumas tecnologias utilizadas.
O Capítulo~\ref{chap:syst} diz respeito à aplicação utilizada neste projeto, a MyDiabetes. Nele descrevemos o estado atual da aplicação, como as funcionalidades que disponibiliza e também as variáveis que permite aos utilizadores registar.
Nos capítulos~\ref{chap:dese} e~\ref{chap:results} apresentamos a nossa contribuição. Discutimos sobre variáveis envolvidas na diabetes, sobre a sua relevância para vários tipos de estudo, sobre os pacientes escolhidos para o estudo e apresentamos uma análise dos dados recolhidos por paciente. Concluímos este trabalho e apresentamos perspetivas de trabalhos futuros no capítulo~\ref{chap:conc}.





