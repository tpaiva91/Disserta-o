\chapter{Introdução}\label{chap:intro}

\section{Contexto}

A diabetes, também conhecida por \textit{diabetes mellitus}, é uma doença crónica bastante comum, conhecida por fazer com que os seus portadores tenham níveis de glicose (açúcar) no sangue mais elevados que o normal. Isto deve-se ao facto de o pâncreas não funcionar da forma devida ou nem sequer funcionar, de todo. Antes de nos aprofundarmos sobre a doença em si, eis alguns factos preocupantes:

Segundo a International Diabetes Federation (IDF), em 2014, cerca de 387 milhões de pessoas tinham diabetes. Em 2035, este número aumentará para 592 milhões e, de acordo com a Organização Mundial da Saúde, em 2030 a diabetes será a sétima causa de morte no planeta. 

Como se pode perceber, esta doença afeta muita gente e a tendência é para piorar. Por isso mesmo, torna-se cada vez mais importante conseguir adiar ou prevenir o seu aparecimento, que nem sempre é possível. O problema é que a diabetes não tem cura e portanto é fundamental que um paciente diabético tenha um tratamento adequado, sendo que o objetivo é manter os níveis de glicemia mais ou menos constantes, e dentro de intervalos considerados normais. 
No entanto, não existe uma forma de tratamento padrão que possa ser aplicada a todos os doentes diabéticos. Além do tratamento médico, como a insulina ou medicamentos, há outros fatores que impactam, de alguma forma, a quantidade de glicose no sangue, como por exemplo o exercício, doenças ou o tipo de alimentos que se ingere. 
Nem toda a gente tem as mesmas rotinas e portanto, um tratamento que seja eficaz num paciente pode não ser noutro. É por isso importante que os pacientes diabéticos tenham um tratamento personalizado, de acordo com as suas características e rotinas. Normalmente, o tratamento de um paciente diabético passa por um plano elaborado conjuntamente pelos seus médicos endocrinologista e nutricionista. Este plano será sempre feito tendo em conta o paciente, pelo que é um plano personalizado de acordo com as necessidades e rotinas do mesmo. Isto é a base de um conceito que será abordado no próximo capítulo, medicina personalizada.

Ainda no tratamento da doença, a parte da alimentação e rotinas é bastante importante. A diferença entre fazer sempre as mesmas refeições a horas certas ou não ter qualquer tipo de rotina neste aspeto pode ser a diferença entre valores normais ou descontrolados. Uma medição frequente, para que o paciente vá controlando os seus níveis de glicemia e tomar ações, se necessário, é um fator importante para a estabilização dos valores de glicose. 
De facto, um controlo apertado dos níveis de glicose pode minimizar ou até prevenir as consequências da diabetes, como vamos ver na próxima secção.



\section{Motivação}

Na última secção mencionámos que o controlo dos níveis da glicose, através de medições frequentes, é um fator importante para o aumento da qualidade de vida do doente diabético. Um estudo levado a cabo entre 1983 e 1993 [controlo.pdf] comprova isto mesmo: participaram 1441 voluntários e nesse período de 10 anos tiveram um controlo intensivo da glucose que lhes permitia ter valores próximos dos normais. O controlo intensivo era feito aumentando o número de medições diárias, aumentando o número de injeções de insulina ou com o uso de bomba, ajustando sempre o valor de insulina de acordo com a comida e exercício, seguindo uma dieta e plano de exercícios e fazer visitas mensais ao centro de saúde para avaliar o progesso. O estudo concluiu que um controlo intensivo da glucose levou a uma redução em pelo menos 50\% de risco de doenças renais, oculares ou do sistema nervoso. 
Ou seja, apesar de ser uma doença crónica, é possível aumentar a qualidade de vida dos pacientes diabéticos, desde que tenham os cuidados acima mencionados. 
Como é possível perceber, a medição e registo da glicose são processos fundamentais para um bom tratamento da doença. No passado, esse registo tinha que ser feito em papel, que tem como inconveniente o facto de ser passível de se perder ou tornar rapidamente confuso e extenso. No entanto, hoje isso já não se verifica.
A tecnologia evoluiu de tal forma que foram criados dispositivos com o propósito de medir e registar os níveis de glicemia. Mas os próprios telemóveis, que são cada vez mais baratos e melhores, tornaram-se inteligentes e são hoje ferramentas poderosas que fazem muito mais do que apenas ligar a alguém ou enviar mensagens. 
Um \textit{smartphone} pode servir para fotografar, jogar ou até ouvir música, mas pode ser usado também como uma ferramenta para o nosso bem-estar, o que se verifica, havendo aplicações destinadas à saúde. A motivação para este trabalho foi a possibilidade de juntar duas áreas diferentes, a saúde e a tecnologia, para desenvolver uma ferramenta que possa ter um impacto positivo na vida dos doentes diabéticos. A próxima secção descreve o projeto com mais detalhe.


\section{Projeto}


Esta dissertação integra-se no projeto "Smart Diabetes Self-Management" que conta com uma aplicação para Android chamada "My Diabetes". Esta aplicação visa oferecer aos seus utilizadores uma alternativa para o registo das medições de glicose, que facilita a visualização desses mesmos registos, através de gráficos ou em forma de lista. A aplicação será descrita mais detalhadamente no capítulo 4. 

O trabalho proposto nesta dissertação foi o de desenvolver novas funcionalidades para a aplicação, dando-lhe alguma "inteligência". Foi proposto, então, desenvolver um sistema que, através da análise dos dados inseridos por cada utilizador ao longo do tempo, fosse capaz de aprender as rotinas para que pudesse gerar avisos ou conselhos face a situações anormais ou até mesmo descobrir padrões que levem a resultados indesejados. Ao descobrir uma destas situações e alertar o utilizador para a mesma, estará a contribuir para que este consiga melhorar o seu controlo da glicemia.

Para desenvolver esta nova funcionalidade, foi necessário obter dados de pacientes insulino-dependentes. Deste modo, em parceria com o Hospital de São João do Porto, foi levada a cabo uma sensibilização dos doentes para utilizarem a aplicação de forma voluntária, sendo que, no futuro, serão estes os maiores beneficiados.
A utilização voluntária da aplicação por parte dos pacientes tem diversos objetivos: 1) obter \textit{feedback} da aplicação em si, como críticas ou sugestões; 2) poder construir \textit{data sets} de registos glicémicos num espaço temporal, algo escasso na \textit{web}. Esta parte de obtenção e análise dos dados é fundamental uma vez que permite ter mais conhecimento do tipo de dados que vão ser analisados, bem como o tipo de padrões ou regras que podem ser descobertas. Desta forma, será possível saber o que é útil ou não, para que a aplicação apenas mostre o que realmente for importante. 

A análise será feita aos dados que os utilizadores inserirem na aplicação e enviarem. Mais informações tais como os dados registados e recolhidos ou o processo de participação no estudo serão abordados com mais detalhe no capítulo 5.


\subsection{Objetivos}

O objetivo final desta dissertação é desenvolver um sistema capaz de gerar regras e mostrar avisos ou conselhos a partir dos dados inseridos, em tempo real e integrá-lo numa aplicação já existente. No entanto há mais objetivos:

\begin{itemize}
	\item Obter dados de registos glicémicos através da participação de voluntários diabéticos;
	\item Fazer diferentes tipos de análises estatísticas sobre esses dados;
	\item Analisar os dados para reconhecimento de padrões ou anomalias;
	\item Criar regras a partir da análise de dados;
	\item Mostar conselhos ou avisos através das regras geradas;
	\item Integrar este sistema na aplicação MyDiabetes.
\end{itemize}

\subsection{Contribuição}





Portanto, a análise será feita aos dados que os utilizadores inserirem na aplicação. Podem ser registados dados tão diferentes como peso, altura, colesterol, exercício ou doenças. Para a descoberta de regras, alguns dados serão mais relevantes que outros, por terem um impacto direto nos valores de glicose, tais como hidratos de carbono, exercício ou doenças, e portanto a análise será feita principalmente com base nos registos destes dados. Isto servirá para dotar a aplicação de "inteligência", isto é, a aplicação ao longo do tempo aprende o que são valores normais e o que podem ser situações anormais. Ao saber que ocorreram situações anormais, também é possível o que as originaram. Uma vez feita a aprendizagem, sempre que situações semelhantes ocorram, a aplicação mostrará um aviso.

Para a aprendizagem serão utilizadas diferentes técnicas de \textit{Data Mining} que gerarão regras. Essas regras serão então traduzidas para Prolog. A linguagem escolhida foi Prolog por se tratar de uma linguagem de programação lógica fortemente associada a inteligência artificial e \textit{machine learning}. Além disso, e ao contrário de outras linguagens, o Prolog é declarativo. Isto significa que um programa em Prolog é expresso em termos de relações, que são definidas por factos e regras.


A diabetes ocorre quando uma pessoa tem valores de glicose no sangue demasiado elevados. A glicose vem dos carbohidratos e é o único monosacarídeo que produz ATP. Resumindo, a glicose produz energia que vai ser utilizada pelas células. Há diferentes razões para que os níveis de glicose no sangue sejam mais elevados que o desejável: a produção de insulina pode não ser suficiente ou, mesmo que seja, as células não respondem à insulina como deveriam. É portanto natural que os níveis de glicose tendam a aumentar depois das refeições. Embora a diabetes esteja relacionada com a insuficiência de insulina no corpo, há diferentes tipos de diabetes. Os principais são:

- Diabetes Mellitus Tipo 1: A diabetes tipo 1, também conhecida como diabetes insulino-dependente ou diabetes juvenil, caracteriza-se pela produção deficiente de insulina pelo pâncreas e requer que o paciente tome doses de insulina diariamente. Normalmente aparece em jovens e é impossível de prevenir. Estima-se que apenas 5\% dos diabéticos tenha este tipo. Entre os sintomas, incluem-se sede insaciável, fome constante ou perda de peso.

- Diabetes Mellitus Tipo 2: A diabetes tipo 2, ou diabetes não-insulino-dependente, ocorre porque as células não usam a insulina de forma adequada. Neste tipo de diabetes, o problema não está necessariamente no pâncreas, mas sim na forma como as células utilizam a insulina. Cerca de 90\% dos diabéticos tem este tipo de diabetes que, normalmente, é associado a um estilo de vida pouco saudável. Por isso mesmo, este tipo de diabetes é frequentemente resultado de excesso de peso ou falta de exercício físico. É comum os doentes de DM2 não necessitarem de tomar insulina e a medicação é feita através de comprimidos. Este tipo de diabetes surge, geralmente, em pessoas mais velhas, mas tem-se vindo a manifestar também em jovens.

- Diabetes gestacional: Este tipo de diabetes pode aparecer durante a gravidez. Caracteriza-se por ter valores de glicose superiores aos normais mas, ainda assim, abaixo dos valores diagnosticados na diabetes. Normalmente, este tipo de diabetes é descoberto nas consultas e não devido a sintomas. Há também o risco de mulheres que sofram deste tipo de diabetes, poderem no futuro sofrer de diabetes tipo 2.

- Diabetes LADA: "Latent Autoimune Diabetes in Adults", traduzido para "diabetes auto-imune latente em adultos". Este tipo de diabetes é considerado uma variação de diabetes tipo 1, embora com uma evolulção mais lenta. No entanto, muitas vezes é erradamente diagnosticado como tipo 2.

Uma vez que esta doença é crónica, ou seja, não há cura, é extremamente importante que os doentes recebam tratamento adequado. No caso dos portadores de diabetes tipo 1 e de uma pequena percentagem de portadores de tipo 2, uma parte fundamental do tratamento passa pela administração de insulina. No entanto, para todos os tipos de diabéticos, o importante para levar uma vida normal, passa pelo registo regular e constante dos níveis de glicose no sangue. Isto porque a falta de eficácio do pâncreas leva a um aumento natural dos níveis de glicose no sangue e, por outro lado, a toma de insulina, quando em quantidades não adequadas, pode ter o efeito contrário e provocar níveis demasiado baixos de açúcar no sangue. Estes dois estados são conhecidos, respetivamente, por hiper e hipoglicemia.

Tanto a hiper como a hipoglicemia são estados que podem fazer parte do dia-a-dia dos diabéticos. Sam ambos de evitar e potencialmente perigosos. A hiperglicemia pode trazer complicações a longo prazo, como doenças cardíacas e renais. Podem fazer parte dos sintomas de hiperglicemia sede intensa, cansaço, dores de cabeça, visão turva, entre outros.

Por sua vez, a hipoglicemia apresenta alguns sintomas tais como tremores, fome, nervosismo, calor, mas não só. Embora menos preocupante a longo prazo do que a hiperglicemia, a curto prazo a hipoglicemia pode ser mais perigosa. Isto porque o cérebro necessita de açúcar para poder funcionar de forma correta e, quando não o tem, pode levar a que o doente perca a consciência. A partir do momento em que o doente desmaia, não consegue contrariar a hipoglicemia, que normalmente se combate pela ingestão imediata de açúcar.

Como se pode perceber, a medição dos níveis de glicose no sangue e também o seu registo, são fatores fundamentais para que um diabético possa ter uma rotina normal. Em plena era da tecnologia, essa medição é feita através de pequenos dispositivos, os glucómetros, que, numa questão de segundos, conseguem dar uma medição precisa dos níveis de açúcar no sangue. O registo é também importante para comparar valores, de forma a que os médicos consigam acompanhar o historial do doente. Também para o registo há dispositivos próprios. No entanto, num período em que os \textit{smartphones} estão cada vez mais em voga, podem ser a ferramente mais natural apra se efetuarem estes registos. É importante para um médico poder observar os valores de glicose dos seus pacientes, que têm de ser registados pelos próprios pacientes. Um registo em papel torna-se rapidamente confuso e até passível de se perder, enquanto que um registo informático traz mais segurança e organização. Embora hajam dispositivos próprios para o registo dos níveis de glicose, isso pode implicar andar sempre com um dispositivo a mais. Por outro lado, como, em princípio, uma pessoa anda sempre com o seu telemóvel, este registo torna-se mais fácil, visto que oferece grande portabilidade e não necessita de nenhum dispositivo extra. É portanto uma vantagem da aliança da tecnologia à saúde. Não é a única, no entanto.

\textit{Data Mining} é uma área da ciência de computadores que permite, através da análise de grandes quantidades de dados, descobrir padrões e regras que uma análise mais simples pode não detetar. A área de \textit{Data Mining} usa diversos métodos de outras áreas, tais como matemática, inteligência artificial e \textit{machine learning}, para tratar, explorar e obter conclusões acerca dos dados. \textit{Data Mining} pode ter diversos fins, como por exemplo,  deteção de anomalias, associação e classificação.

Como se pode perceber pelo nome, deteção de anomalias tem como objetivo a identificação de valores anormais , que podem apenas ser erros mas podem também ter interesse para uma determinada área.

A associação procura relações entre variáveis e pode quantificar essas relações. Um exemplo deste tipo poderia ser a relação entre o emprego de alguém e o seu carro.

Finalmente, a classificação tem como objetivo estudar conjuntos de dados para depois, ao observar novos dados, conseguir classificá-los corretamente. Por exemplo, ao estudar um \textit{dataset} com alguns dados e tendo uma variável "diagnóstico" como sendo "diabético" ou "não-diabético", ao fazer uma aprendizagem e analisar um outro \textit{dataset} com os mesmos tipos de dados, conseguirá, com alguma precisão, classificar corretamente como "diabético" ou "não-diabético".

Nos últimos anos, a área de \textit{Data Mining} tem-se tornado bastante popular e consequentemente bastante usada em diversas áreas, tais como economia, educação e saúde. É precisamente nesta última área que \textit{Data Mining} se pode tornar especialmente útil. Tal como já discutido anteriormente, a diabetes pode provocar oscilações nos valores de glicose no sangue, que podem não ser percetíveis pelos doentes ou, mesmo que sejam, podem não ser tidas como importantes. No entanto, ao fazer aprendizagem sobre os dados dos pacientes, pode ser possível descobrir que afinal essas oscilações podem ter origiem em comportamentos rotineiros, ou seja, fazem parte de um padrão. Um dos objetivos da \textit{Data Mining} é precisamente descobrir padrões. 

Torna-se portanto óbvio que esta área pode ser útil na medicina, em particular para os portadores de diabetes, ao permitir descobrir rotinas que provocam oscilações nos níveis de açúcar no sangue, e que de outra forma poderiam passar despercebidas. Se pensarmos que, ao desborir a causa de tais oscilações, pode descobrir-se a solução, isto ganha ainda mais importância. De forma a perceber melhor de que forma a análise de dados pode melhorar o controlo da diabetes, tomemos como exemplo o seguinte caso: um determinado paciente tem como rotina fazer exercício às segundas-feiras. Às terças-feiras tem sempre níveis de glicose mais baixo. No entanto, pode acontecer que nos outros dias até tenha valores normais e portanto, não se preocupar com esta situação. Contudo, este valor mais baixo pode ser justificado: no dia anterior praticou exercício e não se alimentou devidamente, tendo hipoglicemia na manhã de terça-feira.

Tendo uma ferramenta que seja capaz de analisar dados, descobrir padrões e regras, o doente não só fica mais consciente de erros que pode cometer no controlo da diabetes como também como os combater. 

Algumas referências para se ver a ordenação  \cite{yaacoub2012}. Aqui outra ainda \cite{etsitr102732}.

Deve-se ler Strunk~\cite{strunk2007elements}

\ac{HTTP} é um protocolo como diz em~\cite{test2000} e também na secção~\ref{chap:stat}.

  Temos aqui o Clausen \cite{Clausen2003}. E para ver várias \cite{yaacoub2012, etsitr102732, strunk2007elements}

Deve-se acrescentar os acrónimos no ficheiro \texttt{acros.tex} e ordená-los alfabeticamente nesse ficheiro.


\section{Projeto}

Esta dissertação integra-se no projeto "Smart Diabetes Self-Management" que conta com uma aplicação Android chamada "My Diabetes". De momento, a aplicação permite ao utilizador registar refeições, que podem ser acompanhadas com fotos, registar medições de glicose, definir objetivos para glicemias, registar doenças, exercícios, entre outros. Contudo, permite apenas registar e visualizar os dados inseridos.

No âmbito desta dissertação, propõe-se desenvolver um sistema capaz de analisar dados introduzidos pelos doentes, que consiga detetar padrões e anormalidades nos dados introduzidos, avisando o utilizador. Para o conseguir, pretende-se ampliar as funcionalidades da aplicação "My Diabetes" para que esta consiga "aprender" o que é normal e o que é "anormal", gerando regras que traduzam as anormalidades, com base no \textit{input} obtido.

Para o conseguir, tornou-se necessário obter dados de pacientes com diabetes insulino-dependente. Deste modo, em parceria com o Hospital de São João do Porto, foi levado a cabo uma sensibilização dos doentes para utilizarem a aplicação de forma voluntário, sendo que, em última análise, estes serão os maiores beneficiados.

A utilização da aplicação por parte de voluntários tem diversos objetivos: em primeiro lugar, obter \textit{feedback} da aplicação em si, tal como críticas ou sugestões; em segundo, poder construir \textit{datasets} de dados reais de registos diabéticos, algo escasso na web. A obtenção de dados reais permite ter uma ideia do tipo de dados que a aplicação vai obter por parte de utilizadores, bem como o tipo de padrões que podem ser descobertos ou regras que podem ser geradas.

Portanto, a análise será feita aos dados que os utilizadores inserirem na aplicação. Podem ser registados dados tão diferentes como peso, altura, colesterol, exercício ou doenças. Para a descoberta de regras, alguns dados serão mais relevantes que outros, por terem um impacto direto nos valores de glicose, tais como hidratos de carbono, exercício ou doenças, e portanto a análise será feita principalmente com base nos registos destes dados. Isto servirá para dotar a aplicação de "inteligência", isto é, a aplicação ao longo do tempo aprende o que são valores normais e o que podem ser situações anormais. Ao saber que ocorreram situações anormais, também é possível o que as originaram. Uma vez feita a aprendizagem, sempre que situações semelhantes ocorram, a aplicação mostrará um aviso.

Para a aprendizagem serão utilizadas diferentes técnicas de \textit{Data Mining} que gerarão regras. Essas regras serão então traduzidas para Prolog. A linguagem escolhida foi Prolog por se tratar de uma linguagem de programação lógica fortemente associada a inteligência artificial e \textit{machine learning}. Além disso, e ao contrário de outras linguagens, o Prolog é declarativo. Isto significa que um programa em Prolog é expresso em termos de relações, que são definidas por factos e regras.

\subsection{Objetivos}

O objetivo da dissertação é integrar um sistema capaz de gerar regras e avisos a partir dos dados inseridos. De um modo sumário, os objetivos são:

- Obter dados através da participação de pessoas de diabetes;

- Analisar esses dados para conseguir detetar padrões ou anomalias;

- Criar regras a partir da análise dos dados;

- Integrar o sistema desenvolvido na aplicação para que esta, de forma autónoma, consiga gerar avisos.

[ESTRUTURA DO DOCUMENTO]
\section{Second Section example}
\subsection{SubSection example}
\lipsum[7-10]

