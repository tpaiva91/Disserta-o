

\prefacesection{Abstract}

We live in a time in which technology is present everywhere. Almost everyone has a smartphone nowadays, which can be more powerful than a computer from 10 or 15 years ago. Technology has an important role on society's development and can make a particularly important contribution to medicine. In this case, to diabetes mellitus, which is a disease that there is no cure for and affects hundreds of millions of people. The treatment to diabetes mellitus is therefore fundamental and involves having an intensive control of blood glucose, given that stable values of blood glucose lead to a normal life without future complications. There are already some contributions of technology to this specific condition, with devices to measure blood glucose or devices to administer insulin in a more automated way. Despite this, technology can go even further to help diabetic people.

One key part of diabetic treatment is that it needs to be personalized to each patient. So, in this work we propose to use different data mining techniques to discover possible patterns that can cause oscilating bloog sugar values. These patterns may go unnoticed by the patients which is why it is so important to find them and therefore have a better glycemic control. We have done some individual statistics on diabetic regists from different people to identify different behaviors.

We have also used the apriori algorithm which is an association algorithm, so we can discover diffrent patterns for each user and represent them in rules.

We have done one other kind of individual analysis by using bayesian networks. These networks allow a cause-efect analysis, i.e., change the different parameter's values and see how it affects the other parameters.

With this work  we realised that individual analysis for diabetic people is, in fact, important in a better glycemic control. We have detected different patterns for each user thus corroborating the importance of individual treament, knowing that the discovery of patterns is the first step towards its correction.

\prefacesection{Resumo}

Vivemos numa época em que a tecnologia está presente em todo o lado. Quase toda a gente tem um \textit{smartphone}, que pode ser mais potente que um computador há 10 ou 15 anos. A tecnologia tem um papel importante para o desenvolvimento da sociedade e pode dar uma contribuição especialmente importante à medicina. Neste caso, à \textit{diabetes mellitus}, por ser uma doença sem cura e que afeta centenas de milhões de pessoas. O tratamento para a \textit{diabetes mellitus} passa por um controlo intensivo da quantidade de glicose no sangue, sendo que a estabilidade deste valor em níveis normais pode permitir levar uma vida normal e sem complicações no futuro. Para esta doença em específico, a tecnologia já ajuda de algumas formas, seja em dispositivos para medir a glicemia ou dispositivos para administrar insulina de uma maneira mais automatizada. No entanto, pode ir ainda mais longe.

Uma parte fundamental do tratamento dos diabéticos, é que este tem de ser personalizado para cada paciente. Assim, neste trabalho propôe-se o uso de diferentes técnicas de \textit{data mining} para descobrir possíveis padrões por parte dos doentes diabéticos que possam levar a oscilações de glicemia. Estes padrões podem ser impercetíveis pelos diabéticos, pelo que a sua descoberta é importante para um melhor controlo. Para tal, neste trabalho foram feitas diferentes análises à procura de informação que possa ajudar no controlo da glicemia. Foram feitos algumas estatísticas individuais sobre dados de utilizadores para identificar diferentes comportamentos.

Foi também usado o algoritmo de associação, \textit{apriori}, de forma a descobrir padrões para cada utilizador e representá-los em forma de regras. 

Foi ainda feito um outro tipo de análise individual utilizando redes \textit{bayesianas}. Estas redes permitem uma análise causa-efeito, isto é, alterar os vários valores dos diferentes parâmetros para que se consiga perceber o efeito que isso tem sobre os restantes parâmetros.

Com este trabalho percebeu-se que a análise individualizada para pacientes diabéticos pode ser uma mais-valia para um melhor controlo da glicemia. Foram detetados diferentes padrões para diferentes utilizadores que corroboram esta ideia, sendo que a sua deteção é o primeiro passo para a sua correção.


\prefacesection{Agradecimentos}

A elaboração desta dissertação foi possível graças à ajuda de um conjunto de pessoas, às quais não poderia deixar de agradecer.

Em primeiro lugar, o meu obrigado aos orientadores, Professora Inês Dutra e Professor Pedro Brandão, pela orientação, paciência demonstrada e pela ajuda, fatores determinantes para que possa ter terminado a dissertação.

Ao Dr. Celestino Neves, médico endocrinologista do Hospital de São João, pela sua sempre presente simpatia e ajuda no processo de recolha de dados.

Aos meus amigos, por me proporcionarem alguns dos melhores momentos da minha vida. Costuma dizer-se que a vida académica é a melhor fase da vida, e vocês tentaram que assim fosse. Levo memórias inesquecíveis ao longo destes anos e por isso vos agradeço.

À minha namorada, Catarina Machado. Ao longo destes anos experienciei várias emoções e tu sempre soubeste tirar o melhor de cada uma delas. Quando estava desanimado eras tu que me animavas e quando estava pessimista, tu eras otimista por ambos. Obrigado pelo apoio constante, pela paciência e por estares sempre disponível para me aturares.

Por último, gostaria de agradecer à minha família. Aos meus pais, por ao longo da vida terem colocado sempre as necessidades dos filhos à frente das suas próprias necessidades e possibilidades. Este trabalho não teria sido possível sem a vossa ajuda. Um obrigado também às minhas tias, pela preocupação e ajuda constantes.
