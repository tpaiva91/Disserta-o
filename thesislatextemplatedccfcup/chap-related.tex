\chapter{Estado da Arte}\label{chap:stat}

Neste capítulo, pretende-se analisar com algum nível de detalhe duas vertentes: aplicações para Android para registo de diabetes e o uso de técnicas de \textit{Data Mining} em medicina personalizada. Medicina personalizada é a prática de tratar cada doente de forma individualizada, de acordo com as suas características, necessidades e preferências a cada momento, em vez de um tratamento generalizado para todos os pacientes.  [PersMed.pdf]


\section{Aplicações para registo de diabetes}
Quanto a aplicações para Android de registo de diabetes, existem várias alternativas. Foram escolhidas as cinco aplicações mais populares (ou seja, com mais \textit{downloads}) na \textit{Google Play}. Estas aplicações foram instaladas e testadas afim de perceber aquilo que cada uma oferece ao utilizador. De salientar que foram escolhidas apenas aplicações disponíveis de forma gratuita. 

\subsection{Diário da Diabetes mySugr}

Esta aplicação permite ao utilizador adicionar registos. Cada registo permite especificar alguns parâmetros, como o nível de glicemia, hidratos de carbono consumidos, tipo de insulina e tipo de refeição. Cada registo pode ser acompanhado para uma foto, caso seja uma refeição, e pode ser também escolhido um tipo para cada registo, como por exemplo "almoço", "jantar", "hipoglicemia", entre outros. Para cada registo é ainda possível escolher um outro tipo que dá mais informação, como "Stressado", "Doente", "Álcool", mas não só. De nota também que é possível especificar o tipo de alimentos caso o registo se trate de uma refeição. Entre os tipos de alimentos existem, entre outros, "Legumes", "Carne", "Peixe", "Ovos", etc.

Esta aplicação permite a sincronização com um glicómetro, o "iHealth BG5". [ref] É ainda possível definir metas como limite para hipo e hiperglicemia, e metas de peso ou exercício. Uma característica interessante da aplicação é ter um sistema de pontos e de desafios. Os desafios são diversos, como por exemplo "Caminhada para a cura", que incentiva o utilizador a registar pelo menos 30 minutos de exercícios em 24 horas. Desafios completos desbloqueiam novos desafios. 

Por cada registo efetuado ganha-se uma quantidade de pontos, que é maior quantos mais parâmetros forem preenchidos em cada registo. A aplicação tem um pequeno boneco animado que vai sendo desbloqueado com pontos. Estes dois sistemas são interessantes porque podem funcionar como um incentivo extra para o uso regular da aplicação. 

Por ifm, a aplicação possibilita a exportação dos registos efetuados para três formatos possíveis: xls, pdf ou csv. Esta característica, no entanto, está disponível apenas na versão paga.

\subsection{Diabetes:M}

Esta aplicação permite o registo de glicose, hidratos de carbono consumidos, insulina de efeito rápido e longo, peso, colesterol, pressão arterial, atividade física e hemoglobina glicosilada (também conhecida por hemoglobina glicada ou HbA1c). À primeira vista, nota-se logo o ecrã principal que se pode tornar confuso pela grande quantidade de botões que oferece. As funções disponibilizadas são bastante semelhantes às da aplicação anterior. Uma função nova é a de alarme, que ajuda os utilizadores a não se esquecerem de medir a glicose. Em termos de visualização dos dados inseridos, a aplicação mostra os mesmos em forma de gráficos para se poder acompanhar os registos num determinado intervalo de tempo. É possível verificar que se podem usar unidades de medida diferentes para os vários parâmetros. Por exemplo, para a glicemia pode-se usar mg/dL ou mmol/L. Uma vantagem do ecrã principal é mostrar a quantidade de insulina ativa presente num dado momento. Isto é, se um utilizador tomar 5 doses de insulina, a aplicação mostra, ao longo do tempo, um valor denominado "Insulina Ativa", ou seja, a previsão da insulina que "sobra" desde a última toma.

Uma outra característica interessante é a de possibilitar sincronizaçã com aplicações externas, como Dropbox, Google Drive e Google Fit. A aplicação permite ainda fazer \textit{backup} dos dados.

É também possível exportar e importar dados nos formatos csv e xls, bem como importar dados de glicómetros de diferentes modelos, tais como OneTouch, Dexcom ou Accu-Chek.

\subsection{OnTrack Diabetes}

Esta aplicação permite registar glicose, refeições, exercício, medicação, peso, pressão arterial, pulsação e HbA1c. Tem uma interface bastante simples relativamente às outras aplicações experimentadas. Tem apenas três menus no ecrã principal, que permite ver relatórios, o histórico e alguns gráficos relativamente aos dados inseridos. O ecrã principal mostra também as médias dos níveis de glicose diários, semanais e mensais. Ao explorar a aplicação foi possível verificar que esta oferece vários gráficos. Por exemplo, é possível visualizar, através de gráficos, valores de glicose, média diária de glicose, glicose por hora do dia, exercício, etc.

Ao consultar o menu "Histórico" os dados aparecem na forma de lista e por ordem de refeição, ou seja, para um mesmo dia, os dados relativamente ao pequeno almoço aparecem antes do jantar. Este menu apresenta, portanto, todos os dados registados em cada dia. 
No menu "Relatórios", podemos observar médias de glicose, que são diárias, semanais, mensais ou trimestrais. Existe uma outra opção chamada "glicose por categoria", que mostra os valores médios da glicose registados em cada tipo de refeição.
Uma outra funcionalidade, "Logbook", permite a visualização dos dados através de gráficos, permitindo ver qualquer parâmetro registado e partilhar esses mesmos gráficos por \textit{e-mail}.

É possível exportar os dados para csv, xml ou html. É também possível criar \textit{backup} ou apagar todos os dados num determinado intervalo de tempo.

\subsection{Diabetes - Diário Glucose}

De todas as aplicações analisadas, esta é a mais simples. É a que menos funções oferece, permitindo registar apenas o peso e a glicose, que é feito no ecrã principal. A aplicação é composta por outros três separadores que permitem visualizar os níveis de glicose em lista e em gráfico. É possível exportar os dados registados para um ficheiro pdf ou partilhar por \textit{e-mail}.

\subsection{Glucose Buddy: Diabetes Log}

Esta aplicação permite registar o tipo de diabetes, peso, altura, pressão arterial, glicose, HbA1c, exercício, refeições e a atividade do registo (refeição, antes de exercício, depois de exercício, etc.).

Pode-se observar os registos de glicose em forma de lista, utilizando o menu "Logs" ou em forma de gráfico usando o menu "Graphs". No gráfico pode-se visualizar apenas o parâmetro da glicose bem como a média de todos os valores registados por dia.

A aplicação oferece ainda um alarme que pode ser ativado para uma determinada hora ou então pode ser coordenado com um evento. Por exemplo, o utilizador pode definir um alarme para 30 minutos depois do almoço, sendo que quando fizer um registo com o tipo de refeição "almoço", ativará o alarme para o tempo definido. 

É possível exportar os registos selecionando intervalos pré-estabelecidos pela aplicação e enviar para o \textit{e-mail}.\newline





Como se pode perceber, as aplicações não diferem muito entre si e todas elas oferecem praticamente as mesmas funcionalidades. 


\section{Data Mining na saúde}


Nesta secção pretende-se abordar de que forma a área de \textit{Data Mining} pode ser útil para a saúde. Vamos analisar algum trabalho feito na área da saúde utilizando técnicas de \textit{Data Mining}, de uma forma geral, e também o que já foi feito em específico para a diabetes. 

Estas técnicas podem ser utilizadas para fins diferentes: fazer aprendizagem analisando dados já existentes para que se possam criar modelos, que por sua vez irão classificar novos dados; encontrar relações entre variáveis e causas; detetar padrões.

Em [tiago.pdf], os autores usaram diferentes algoritmos para tentar prever a sobrevivência ao cancro da mama. Neste caso, define-se por sobrevivência o paciente estar vivo pelo menos 5 anos após o diagnóstico do cancro. Foram usados três algoritmos de classificação diferentes: redes neuronais artificiais, árvores de decisão e regressão logística. Os autores usaram um \textit{dataset} já existente e, depois de todo o pré-processamento, como limpeza de dados, obtiveram um \textit{dataset} com 17 variáveis (16 variáveis de previsão e 1 variável de classe, isto é, a variável a ser prevista). Gerando modelos através dos três algoritmos utilizados, conseguiram classificar, com alta percentagem de precisão, se um dado paciente teria sobrevivido ou não. Além disso, conseguiram também descobrir quais as variáveis mais importantes para a classificação, e, portanto, atribuir importâncias diferentes a diferentes variáveis.

Num outro estudo, conduzido por Soni \textit{et. al.} [soni.pdf]




