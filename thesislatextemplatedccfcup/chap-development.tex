\chapter{Desenvolvimento}\label{chap:dese}

\begin{lstlisting}[numbers=none,language=java,caption={[CommandDaemonCallsItf]
   {CommandDaemon} callback interfaces},label=lis:commandDCallsItfs,float=htb]
public interface CallBackCmdMeasurements { // comment
	public abstract void newMeasure(MeasurementBasic measure, int reqId);
	public abstract void newMeasuresAggSimp(MeasurementBasic[] measuresAggSimp, 'A string');
}
\end{lstlisting}

É possível como referir o código, por exemplo o bloco de código~\ref{lis:commandDCallsItfs}.

Outro acrónimo pode ser \ac{TCP} (que deve estar expandido aqui, apesar de ter sido usado já no capítulo \ref{chap:syst}.

Podem ver a figura \ref{fig:logoFCUP} muito bem. Notem que na lista de figuras nao aparece tudo o que está na caption, mas só o que está entre [].



\begin{figure}[htb]
   \centering % center the figure
   \includegraphics[scale=.4]{pics/fc_logo}
   \caption[FCUP logo velho]{O logo da FCUP antigo}\label{fig:logoFCUP}
\end{figure}


\lipsum[3-7]
\section{Teste}
\lipsum[1]
E isto é o fim do cap. \ref{chap:dese}.

