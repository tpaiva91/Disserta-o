\chapter{Conclusões}\label{chap:conc}

Como fomos vendo ao longo desta dissertação, a \textit{diabetes mellitus} é uma doença que afeta milhões de pessoas e a tendência é que este número aumente. Vimos também que não há cura e que a qualidade de vida dos doentes diabéticos pode ser mantida através de um tratamento adequado. Esse tratamento é personalizado para cada doente e é feito através de medicamentos e também de um estilo de vida saudável. Embora os medicamentos sejam imprescindíveis, o estilo de vida também assume uma importância bastante grande: um estilo de vida saudável pode fazer com que esta doença nunca apresente consequências e, por outro lado, um estilo de vida com maus hábitos pode levar a consequências bastante graves. 

Neste trabalho, estes factos foram tidos em conta e fez-se uma análise personalizada para registos de alguns utilizadores. O objetivo desta análise era perceber de que forma é que os vários parâmetros inerentes a uma pessoa diabética, assim como as suas rotinas, podiam provocar oscilações nos valores de glicemia. 

Este trabalho permitiu 



\section{Trabalho Futuro}\label{sec:trab}


