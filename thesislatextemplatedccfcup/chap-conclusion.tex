\chapter{Conclusões}\label{chap:conc}

Ao longo desta dissertação fomos enfatizando duas coisas acerca da diabetes: é uma doença sem cura e, com tratamento adequado, os doentes diabéticos podem levar uma vida normal. Vimos também que não há um tratamento generalizado e que cada doente tem de ter um tratamento personalizado. Assim sendo, propusemos fazer diversos tipos de análise para um conjunto de utilizadores. Estas análises foram personalizadas para cada utilizador para tentar perceber que tipo de comportamentos é que os doentes diabéticos têm que provocam alterações na glicemia. Esta análise permitiu perceber que, de facto, diferentes utilizadores têm diferentes rotinas e hábitos e, portanto, incorrem em diferentes padrões. Na parte de regras de associação foi possível perceber que diferentes utilizadores tinham diferentes regras. Embora as regras possam ser do mesmo tipo, a parte importante é que são diferentes, evidenciando, uma vez mais, a importância de um tratamento individualizado. Também na parte das redes \textit{bayesianas} foi possível perceber esta necessidade: para cada utilizador foi gerada uma rede diferente e as variáveis relacionavam-se de maneira diferente em cada uma das redes. 
Foi ainda possível concluir a viabilidade do uso de \textit{data mining} em registos de doentes diabéticos: conseguimos gerar um número aceitável de regras diferentes com uma amostra de utilizadores pequena, o que faz prever que, para um conjunto bastante maior de dados, vão ser descobertos ainda outros tipos de padrões e portanto, auementar a variedade de regras. 



\section{Trabalho Futuro}\label{sec:trab}


