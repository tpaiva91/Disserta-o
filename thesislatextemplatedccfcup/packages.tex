%%%%%%%%%%%%%%%%%%%%%%%%%%%%%%%%%%%%%%%%%%%%%%%%%%%%%%%%%%%%
% Pedro Brandão adaptation from UPThesis of Fernando Silva
% (see upthesis.sty)
% Added more packages and configuration
% Should be used with pdflatex for better output
%
% 2014/07/08 First draft
% 2014/07/21 added the List of code blocks and their ``name'' for portuguese
%        this includes the caption name
%
% TODO there should be many...
% example add the required cover page: current solution is
% to print it using word to pdf and then join the pdfs
%%%%%%%%%%%%%%%%%%%%%%%%%%%%%%%%%%%%%%%%%%%%%%%%%%%%%%%%%%%%

% Particular settings for the layout of the thesis, including the old FC logo
% The header format is also on that file
\usepackage{upthesis}

% Encoding
\usepackage[utf8]{inputenc}
% see http://tex.stackexchange.com/questions/664/why-should-i-use-usepackaget1fontenc
\usepackage[T1]{fontenc}


%image placement
\usepackage{float}
\usepackage{graphicx}

% Use colour names
\usepackage[usenames,dvipsnames]{color}
% for named colors (see 
% http://en.wikibooks.org/wiki/LaTeX/Colors#The_68_standard_colors_known_to_dvips)
\definecolor{light-gray}{gray}{0.94}
\definecolor{gray-cell}{gray}{.80}
\usepackage{colortbl}
\usepackage{tabularx}
\usepackage{multicol}
\usepackage{multirow}


% rotating introduces some glitches with acronym. Acronyms used in the sidewaystable were marked as being multiply
% defined. They were expanded the 1st time on the table. Using \acs solved the issue
\usepackage{rotating}

% used for code examples and other listings
\usepackage{listings}
\lstset{
  aboveskip={\baselineskip},
  % belowskip={1.5\baselineskip},
 % columns=fixed,
 % extendedchars=true,
  % prebreak = \raisebox{0ex}[0ex][0ex]{\ensuremath{\hookleftarrow}},
  %identifierstyle=\bf\ttfamily,
  keywordstyle=\color{Maroon}\bf\ttfamily,
  commentstyle=\normalfont\color{BlueViolet}\itshape,
  stringstyle=\color{OliveGreen},
%  lineskip=-2.0pt,
  %from http://en.wikibooks.org/wiki/LaTeX/Packages/Listings
%  language=Octave,                % choose the language of the code
  basicstyle=\footnotesize\ttfamily,       % the size of the fonts that are used for the code
%  numbers=left,                   % where to put the line-numbers
%  numberstyle=\footnotesize,      % the size of the fonts that are used for the line-numbers
%  stepnumber=2,                   % the step between two line-numbers. If it's 1 each line 
%                                  % will be numbered
  numbersep=5pt,                  % how far the line-numbers are from the code
  backgroundcolor=\color{light-gray},  % choose the background color. You must 
                                       %add \usepackage{color}
%  showspaces=false,               % show spaces adding particular underscores
  showstringspaces=false,         % underline spaces within strings
%  showtabs=false,                 % show tabs within strings adding particular underscores
%  frame=shadowbox,
  frame=single,	                % adds a frame around the code
  frameround=tttt,	                % t or f(alse) for each corner
  framesep = 1pt,
  %framexleftmargin = -10pt,
  fillcolor=\color{white},
  rulecolor=\color{MidnightBlue},
  %rulesepcolor=\color{blue},
  tabsize=3,	                % sets default tabsize to 3 spaces
  captionpos=b,                   % sets the caption-position to bottom
  xleftmargin = 12pt,
  xrightmargin = 10pt,
  breaklines=true,                % sets automatic line breaking
  breakatwhitespace=true,        % sets if automatic breaks should only happen at whitespace
%  title=\lstname,                 % show the filename of files included with \lstinputlisting;
%                                  % also try caption instead of title
%  escapeinside={\%*}{*)},         % if you want to add a comment within your code
%  morekeywords={*,...}            % if you want to add more keywords to the set
   escapechar=\#,                   % enable to right latex and be interpreted as such
}

\makeatletter
  \@ifpackagewith{babel}{portuguese}{%
    \addto\captionsportuguese{\renewcommand*\lstlistlistingname{Lista de Blocos de Código}}%
    \addto\captionsportuguese{\renewcommand*\lstlistingname{Bloco de Código}}%
  }{}
\makeatother

% This adds the possibility of having lists in paragraphs
\usepackage{paralist}

% Enhance/improve fonts, kerning and spacing
\usepackage{microtype}
% Need to be a font compatible with microtype (or remove microtype package)
\usepackage{lmodern} %% See for fonts http://www.tug.dk/FontCatalogue/ and http://www.tug.org/fonts/
%fonts examples: newcent, utopia, charter, concmath (may need cm-super)
% can also add the cm-super package (don't usepackage on the tex file,
% just install using the package manager)

% for monotype use courier
% http://www.macfreek.nl/mindmaster/LaTeX_Bold_Typewriter_Font
\usepackage{courier}
% fix the restriction for font-sizes with latex
% see http://www.tex.ac.uk/cgi-bin/texfaq2html?label=fontunavail
\usepackage{fix-cm}

% Math fonts
\usepackage{amsfonts}
\usepackage{amsmath}

% Bibliography
% for natbib see http://en.wikibooks.org/wiki/LaTeX/More_Bibliographies#Options
%\usepackage[authoryear,sort]{natbib}
\usepackage[numbers,sort&compress]{natbib}

% Acronyms package
\usepackage{acronym}

% Enable multicoluns, used in the acronyms appendix
\usepackage{multicol}

% just for testing purposes (get ipsum text)
\usepackage{lipsum}

% URLs, linking, and pdf parameters
% break long urls, should be before hyperref
\usepackage[hyphens]{url}

%%%%%%%%%%%%%%%%%%%%%%%%%%%%%%%%%%%%%%%%%%%%%%%%%%
%%% Headers/Footers/Chapters
%%%%%%%%%%%%%%%%%%%%%%%%%%%%%%%%%%%%%%%%%%%%%%%%%%
% See the upthesis for that
