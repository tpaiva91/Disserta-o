\chapter{MyDiabetes}\label{chap:syst}

Neste capítulo vamos analisar com mais detalhe a aplicação utilizada neste projeto de dissertação. A aplicação chama-se \textit{MyDiabetes} e foi desenvolvida no âmbito do mesmo projeto em que esta dissertação se insere. 
Por não haver \textit{data sets} existentes com o formato e tipo de dados pretendidos, tornou-se necessário recolhê-los. Para tal, disponibilizámos a aplicação para ser utilizada de forma voluntário por alguns utilizadores diabéticos. Este processo será explicado no próximo capítulo.
Esta aplicação foi a escolhida para disponibilizar aos voluntários não só por ter sido desenvolvida neste mesmo projeto mas também pelo maior controlo que permite ter ao permitir registar todos os dados que possam ser relevantes para esta dissertação. 
Outra característica que a torna relevante é o facto de ter um sistema que permite o envio dos dados dos utilizadores para o projeto, para que estes possam ser utilizados. Sendo então esta a aplicação utilizada, vamos descrever alguns aspetos importantes.


\section{Objetivo da aplicação}

O objetivo desta aplicação já foi mencionado anteriormente: ajudar o doente diabético, ao oferecer uma ferramenta alternativa que permita registar e visualizar todos os parâmetros importantes, como glicose, insulina e hidratos de carbono. O \textit{smartphone} é um dispositivo bastante interessante para ter aplicações como esta: a maioria das pessoas tem um e portanto, se tiver uma aplicação pode registar a glicemia a qualquer hora e em qualquer lugar.
Além da função de registo, a aplicação permite também a visualização dos registos efetuados em forma de gráfico. Assim torna-se mais fácil detetar hiperglicemias, por exemplo. Ou visualizar registos de meses anteriores ou até mesmo mostrar ao médico, durante a consulta, os valores de glicemia no período entre as consultas.
No futuro, a aplicação será mais interativa com o utilizador ao mostrar diferentes avisos ou conselhos. Tudo isto contribui para o objetivo principal: tornar mais simples e eficaz o controlo da glicemia. 

Para esta dissertação, o objetivo foi o de familiarizar os utilizadores com a aplicação e também servir de ferramenta para enviar os dados necessários.

\section{Arquitetura}


A aplicação é bastante intuitiva: tem menus de registo, como registo de refeições, insulinas, exercício ou doenças. Esses registos são guardados no \textit{smartphone}, numa base de dados \textit{sqlite}. Tem também um menu para visualização dos registos efetuados, o "Logbook". 

\section{Variáveis recolhidas}

A aplicação permite recolher dados sobre doentes diabéticos através dos registos que estes vão efetuando. Além de valores de glicose, hidratos de carbono e insulina, como já mencionado anteriormente, permite também a recolha de outros dados como doença ou exercício. A data e hora de cada registo também são guardadas, isto porque pode ser útil descobrir padrões temporais, associados a um dia da semana ou hora do dia, por exemplo. Permite ainda recolher outras variáveis como pressão arterial, colesterol e peso. Estas variáveis não são tão relevantes, no entanto, por não haver uma relação direta entre estas e os valores de glicose. A recolha destas variáveis é feita através de uma funcionalidade presente na aplicação, que permite ao utilizador mandar a base de dados ao gestor do projeto. 





